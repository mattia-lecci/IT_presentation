\section{Gaussian Channel with Deterministic $H$}
%%%%%%%%%%%%%%%%%%%%%%%%%%%%%%%%%%%%%%%%%%%%%%%%%%%%%%%%%%%%%%%%%%%
%%%%%%%%%%%%%%%%%%%%%%%%%%%%%%%%%%%%%%%%%%%%%%%%%%%%%%%%%%%%%%%%%%%
\subsection{Capacity Derivation}
\begin{frame}[allowframebreaks]{Capacity Derivation}
We first consider the case of a deterministic (fixed) transfer function $H$. Note that the capacity is a function of both the transfer function and the power constraint, i.e. $C(H,P)$. We consider the case where both receiver and transmitter know the matrix $H$.

\begin{block}{Singular Value Decomposition (SVD)}\justify
Any matrix $H\in \C{r\times t}$ can be decomposed as
$$H = UD\herm{V}$$
where $U\in \C{r\times r}$ and $V \in \C{t\times t}$ are unitary, $D\in \C{r\times t}$ is diagonal with non-negative real entries.

In fact, the columns of $U$ are the eigenvectors of $H\herm{H}$, the columns of $V$ are the eigenvectors of $\herm{H}H$ and the diagonal entries of $D$, called \textit{singular values} of $H$, are the eigenvalues, which coincide for the two cases and given the hermitianity of such matrices, they are real and non-negative.
\end{block}

%%%%%%%%%%%%%%%%%%%%%%%%%%%%%%%%%%%%%%%%%%%%%%%%%%%%%%%%%%%%%%%%%%%
\framebreak

Thus, the problem can be seen as follows
$$\vb{y} = UD\herm{V}\vb{x} + \vb{n}$$

Therefore, preprocessing the transmitted symbol as $\vb{x} = V\vt{x}$ and $\vt{y}=\herm{U}\vb{y}$ and defining $\vt{n} = \herm{U}\vb{n}$, we get
$$\vt{y} = D\vt{x} + \vt{n}$$
as an equivalent channel.

\myspace
Calling $\sigma_i = \lambda_i^{\frac{1}{2}}$ the singular values for $i=1,\ldots,\min(r,t)$, we have
\begin{equation*}
\begin{cases}
\tilde{\vb{y}}_i = \sigma_i \tilde{\vb{x}}_i + \tilde{\vb{n}}_i & 1\leq i\leq \min(r,t)\\
\tilde{\vb{y}}_i = \tilde{\vb{n}}_i & i > \min(r,t)
\end{cases}
\end{equation*}

%%%%%%%%%%%%%%%%%%%%%%%%%%%%%%%%%%%%%%%%%%%%%%%%%%%%%%%%%%%%%%%%%%%
\framebreak

It's clear that in order to maximize the capacity we need to choose $\vt{\vb{x}}_i$ for $1\leq i \leq \min(r,t)$ (actually only up to the number of non-zero singular values) to be independent, zero-mean, \cscg.

\medskip
It can be shown that the optimal variances must follow once again the "water-filling" equations
\begin{equation*}
\begin{cases}
\Exp{\Re(\vt{\vb{x}}_i)} = \Exp{\Im(\vt{\vb{x}}_i)} = \frac{1}{2} \qty(\mu-\lambda_i^{-1})^+\\
P = \sum_i \qty(\mu-\lambda_i^{-1})^+\\
C = \sum_i \qty(\log(\mu\lambda_i))^+
\end{cases}
\end{equation*}

\end{frame}

%%%%%%%%%%%%%%%%%%%%%%%%%%%%%%%%%%%%%%%%%%%%%%%%%%%%%%%%%%%%%%%%%%%
%%%%%%%%%%%%%%%%%%%%%%%%%%%%%%%%%%%%%%%%%%%%%%%%%%%%%%%%%%%%%%%%%%%
\subsection{Examples}
\begin{frame}
\begin{example}{Example 1}
	MAYBE??????
\end{example}
\end{frame}

%%%%%%%%%%%%%%%%%%%%%%%%%%%%%%%%%%%%%%%%%%%%%%%%%%%%%%%%%%%%%%%%%%%
%%%%%%%%%%%%%%%%%%%%%%%%%%%%%%%%%%%%%%%%%%%%%%%%%%%%%%%%%%%%%%%%%%%
\subsection{Formal Derivation of Capacity}
\begin{frame}[allowframebreaks]{Formal Derivation of Capacity}

I will now proceed in the derivation of the capacity in a more formal way. This also will be used with some slight differences in the derivation of the capacity when $\vb{H}$ is not deterministic.

\medskip
Recall, the mutual information that we are interested in is
$$I(\vb{x};\vb{y}) = h(\vb{y})-h(\vb{y}|\vb{x}) = h(\vb{y})-h(\vb{n})$$

Note that if $\vb{x}$ satisfies $\Exp{\herm{\vb{x}}\vb{x}} \leq P$, so does $\vb{x}-\Exp{\vb{x}}$, thus we will only focus on zero-mean $\vb{x}$.

\medskip
Furthermore, if $\vb{x}$ is zero-mean with covariance $\Exp{\vb{x}\herm{\vb{x}}} = Q$, then $\vb{y}$ is zero-mean with covariance $\Exp{\vb{y}\herm{\vb{y}}} = HQ\herm{H} + I_r$, and by Lemma 2 among such $\vb{y}$ the entropy is largest when $\vb{y}$ is \cscg, which is the case when $\vb{x}$ is \cscg{} (Lemmas 3 and 4).

%%%%%%%%%%%%%%%%%%%%%%%%%%%%%%%%%%%%%%%%%%%%%%%%%%%%%%%%%%%%%%%%%%%
\framebreak

Considering, then, $\vb{x}$ \cscg, we have
\begin{align*}
\begin{split}
I(\vb{x;y}) &= h(\vb{y})-h(\vb{n}) = \log\det(\pi e(I_r + HQ\herm{H})) - \log\det(\pi e I_r)\\
&= \log\det(I_r + HQ\herm{H}) = \log\det(I_t + Q\herm{H}H)
\end{split}
\end{align*}
where the last equality is given by \textit{Sylvester's determinant identity}, that states $\det(I+AB) = \det(I+BA)$.

\medskip
Since it occurs many times during the paper, we define
$$\Psi(Q,H) = \log\det(I_r + HQ\herm{H})$$

%%%%%%%%%%%%%%%%%%%%%%%%%%%%%%%%%%%%%%%%%%%%%%%%%%%%%%%%%%%%%%%%%%%
\framebreak

Since we're dealing with zero-mean \cscg, we only need to find a positive semi-definite matrix $Q$ to fully determine the distribution, such that $\tr(Q)\leq P$.

\medskip
Notice that $\herm{H}H$ is hermitian, thus it can be decomposed as $\herm{H}H = \herm{V}\Lambda V$ with unitary $V$ and diagonal positive semi-definite $\Lambda$. Thus, using Sylvester's identity again we obtain
\begin{align*}
\begin{split}
\log\det(I_t + Q\herm{H}H) &=
\log\det(I_t + \Lambda^\frac{1}{2}VQ\herm{V}\Lambda^\frac{1}{2})\\
&= \log\det(I_t + \Lambda^\frac{1}{2}\tilde{Q}\Lambda^\frac{1}{2})
\end{split}
\end{align*}
where we noticed that $Q$ and $\tilde{Q}$ are equivalent for both definiteness and power constraint, so we can as well optimize over $\tilde{Q}$.

%%%%%%%%%%%%%%%%%%%%%%%%%%%%%%%%%%%%%%%%%%%%%%%%%%%%%%%%%%%%%%%%%%%
\framebreak

A special case of \textit{Hadamard's inequality} for positive semi-definite matrices, states that
$$I(\vb{x;y})
= \log\det(I_t + \Lambda^\frac{1}{2}\tilde{Q}\Lambda^\frac{1}{2})
\leq \prod_i (1+ \tilde{Q}_{ii}\lambda_i) $$
with equality for diagonal matrices.

Thus, we see that the maximizing $\tilde{Q} = VQ\herm{V}$ is diagonal, and the optimal entries follow the water-filling equations, i.e.
\begin{equation*}
\begin{cases}
\tilde{Q}_{ii} = (\mu-\lambda_i^{-1})^+\\
P = \sum_i \tilde{Q}_{ii}\\
C = \sum_i (\log(\mu\lambda_i))^+
\end{cases}
\end{equation*}

\end{frame}