\section{Introduction}

\subsection{Abstract}
%%%%%%%%%%%%%%%%%%%%%%%%%%%%%%%%%%%%%%%%%%%%%%%%%%%%%%%%%
\begin{frame}{Abstract}
In this presentation I will talk about the capacity of a single-user Gaussian channel with multiple receiving and/or transmitting antennas (also known as \alert{MIMO} channel).

\myspace
I will talk about 3 cases:
\begin{itemize}
	\item Deterministic channel
	\item Random i.i.d. ergodic channel
	\item \textit{Bonus}: multi-user case
\end{itemize}

\end{frame}

%%%%%%%%%%%%%%%%%%%%%%%%%%%%%%%%%%%%%%%%%%%%%%%%%%%%%%%%%%
\subsection{Notation and Assumptions}
\begin{frame}[allowframebreaks]{Notation}
The notation adopted is halfway between the one used during the course and the one from the original paper.

\myspace
Let's denote with $t$ the number of transmitting and with $r$ the number of receiving antennas.

\myspace
We will consider the classical linear model, where $\vb{x}\in \C{t}$ is the transmitted vector and $\vb{y}\in \C{r}$ is the received vector, $H\in \C{r\times t}$ is the complex channel matrix and $\vb{n}\in \C{r}$ is the noise
$$\vb{y} = H\vb{x} + \vb{n}$$

%%%%%%%%%%%%%%%%%%%%%%%%%%%%%%%%%%%%%%%%%%%%%%%%%%%%%%%%%%
\framebreak

We assume noise at different receivers to be independent and normalized, i.e. $\Exp{\vb{x} \herm{\vb{x}}} = I_r$.\\
The dag (\dag) notation is used for the conjugate-transpose operation.

\myspace
We have the power constraint
$$\Exp{\herm{\vb{x}}\vb{x}} = \Exp{\tr(\vb{x}\herm{\vb{x}})} = \tr(\Exp{\vb{x}\herm{\vb{x}}})\leq P$$

\myspace
A complex random vector (r.ve.) $\vb{x}\in \C{n}$ is said to be Complex Gaussian if its real extension $\vu{x} \triangleq \qty[\begin{array}{c}
\Re{\vb{x}}\\
\Im{\vb{x}}
\end{array}]
\in \R{2n}$ is Gaussian.

\myspace
The r.ve. $\vb{x}$ will have \textit{mean} and \textit{covariance} respectively $\mu= \Exp{\vb{x}}$ and $Q = \Cov{Q} = \Exp{(\vb{x}-\mu)\herm{(\vb{x}-\mu)}}$

%%%%%%%%%%%%%%%%%%%%%%%%%%%%%%%%%%%%%%%%%%%%%%%%%%%%%%%%%%
\framebreak

Defining for a complex matrix $A$
$$\hat{A} = \qty[\begin{array}{lr}
\Re(A) & -\Im(A)\\
\Im(A) & \Re(A)
\end{array}]$$
we say that a Complex Gaussian r.ve. is \textit{circularly symmetric} if $\Cov{\vu{x}} = \Exp{(\vu{x}-\hat{\mu})(\vu{x}-\hat{\mu})^T} = \frac{1}{2} \hat{Q}$

\begin{block}{pdf of circularly symmetric Complex Gaussian}
	The pdf of a circularly symmetric Complex Gaussian is
	\begin{align*}
	\begin{split}
	\gamma_{\mu,Q} &= \det(\pi\hat{Q})^{-\frac{1}{2}} e^{-(\hat{x}-\hat{\mu})^T\hat{Q}(\hat{x}-\hat{\mu})}\\
	&= \det(\pi Q)^{-1} e^{-\herm{(x-\mu)}Q(x-\mu)}
	\end{split}
	\end{align*}
\end{block}

\end{frame}

%%%%%%%%%%%%%%%%%%%%%%%%%%%%%%%%%%%%%%%%%%%%%%%%%%%%%%%%%%
\subsection{Properties and Lemmas}
\begin{frame}[allowframebreaks]{Properties and Lemmas}
\begin{block}{Lemma 1}
	The following properties hold:
	\begin{subequations}
	\begin{gather}
	C=AB \iff \hat{C} = \hat{A}\hat{B}\\
	C = A+B \iff \hat{C} = \hat{A} + \hat{B}\\
	C = \herm{A} \iff \hat{C}=\hat{A}^T\\
	C=A^{-1} \iff \hat{C} = \hat{A}^{-1}\\
	\det(\hat{A}) = |\det(A)|^2 = \det(A \herm{A})\\
	z = x+y \iff \hat{z} = \hat{x}+\hat{y}\\
	y=Ax \iff \hat{y} = \hat{A}\hat{x}\\
	\Re{\herm{x}y} = \hat{a}^T\hat{y}
	\end{gather}
	\end{subequations}
\end{block}

%%%%%%%%%%%%%%%%%%%%%%%%%%%%%%%%%%%%%%%%%%%%%%%%%%%%%%%%%%
\framebreak

\begin{block}{Corollary 1}
A matrix $U\in \C{n\times n}$ is unitary if and only if $\hat{U}\in \R{2n\times 2n}$ is orthonormal.
\end{block}
\begin{block}{Corollary 2}
If $Q\in \C{n\times n}$ is positive semi-definite, then so is $\hat{Q}\in \R{2n\times 2n}$.
\end{block}
\begin{alertblock}{Lemma 2}
Suppose the complex r.ve. $\vb{x}\in \C{n}$ is
zero-mean and satisfies $\Exp{x\herm{x}}=Q$. Then the differential entropy of $\vb{x}$ satisfies $h(\vb{x}) \leq \log\det(\pi e Q)$ if and only if $\vb{x}$ is circularly symmetric Complex Gaussian with
$$\Exp{x\herm{x}} = Q$$
\end{alertblock}
In other words, circularly symmetric Complex Gaussian r.ve. are entropy maximizers for the class of complex random vectors.

%%%%%%%%%%%%%%%%%%%%%%%%%%%%%%%%%%%%%%%%%%%%%%%%%%%%%%%%%%
\framebreak

\begin{block}{Lemma 3}
lf $\vb{x}\in \C{n}$ is a circularly symmetric Complex Gaussian then so is $y = Ax$ for any $A \in \C{m\times n}$.
\end{block}
\begin{block}{Lemma 4}\label{lemma:4}
	f $\vb{x}$ and $\vb{y}$ are independent {\cscg s}, then $\vb{z}=\vb{x}+\vb{y}$ is also \cscg.
\end{block}

\end{frame}

%%%%%%%%%%%%%%%%%%%%%%%%%%%%%%%%%%%%%%%%%%%%%%%%%%%%%%%%%%
%%%%%%%%%%%%%%%%%%%%%%%%%%%%%%%%%%%%%%%%%%%%%%%%%%%%%%%%%%