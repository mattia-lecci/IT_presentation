\section{\protect\textit{Bonus}: Multiaccess Channel}
\begin{frame}[allowframebreaks]{\textit{Bonus}: Multiaccess Channel}
Consider now a number of transmitters, say $K$, each with $t$ transmitting antennas, and each subject to a power constraint $P$. There is a single receiver with $r$ antennas.

\medskip
The received signal $\vb{y}$ is given by
$$\vb{y} = \qty[\vb{H}_1 \ldots \vb{H}_K]
\begin{bmatrix}
\vb{x}_1 \\ 
\vdots \\ 
\vb{x}_K
\end{bmatrix} 
+\vb{n}$$
where $\vb{x}_k$ is the signal sent by the $k$-th transmitter, $\vb{n}$ is independent Gaussian noise, and $\vb{H}_k\in \C{r\times t}$.

%%%%%%%%%%%%%%%%%%%%%%%%%%%%%%%%%%%%%%%%%%%%%%%%%%%%%%%%%%%%%%%%%%%
\framebreak

We assume, as for the ergodic case, that the receiver knows all the matrices $\vb{H}_k$, and their entries are i.i.d. \cscg{} with zero-mean and unit variance.

\medskip
Since we know that the optimal single user transmission yields an i.i.d. solution for each antenna, it doesn't matter if the transmitters in the multiuser scenario cannot cooperate.

\medskip
Thus, a rate vector $(R_1,\ldots,R_K)$ will be achievable if
$$\sum_{i=1}^{k} R_{[i]} \leq C(r,kt,kP) \qquad \text{for all } k=1,\ldots,K$$
where $\qty(R_{[1]},\ldots,R_{[K]})$ is the ordering of the rate vector from the largest to the smallest, and $C(a, b, P)$ is the single user
$a$ receiver $b$ transmitter capacity under power constraint $P$.

\end{frame}